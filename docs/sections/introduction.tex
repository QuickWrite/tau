\subsection{What is Tau?}
Tau is a small programming language that intends to implement a small Turing Machine.

It interprets the code by parsing it, compiling it into a graph and then executing it.

\subsection{Motivation}
This programming language is mainly intended for educational purposes. 
It can be used to showcase the abilities and possibilities of a Turing Machine.

Instead of just learning about Turing Machines in an abstract way, 
people can play around and actually see what the different elements of a Turing Machine do.

\subsection{Turing Machine}
A Turing Machine is a conceptual model of computation that was proposed by the British mathematician and logician Alan Turing in 1936. 
This abstract device helps to explore the boundaries of what can be computed and offers a basis for examining algorithms and computational methods. 
Although it doesn't exist as a tangible machine, the Turing Machine is a cornerstone idea in the fields of computer science and mathematical logic.

The main idea is to build a machine as primitive as possible with the least possible things whilst not sacrificing the possibility
to implement any computer algorithm possible. Because of this primitive nature, many programming languages, to show that they can implement any
computer algorithm, test if they are ``Turing Complete''.

\subsubsection{Components}
The different components of a Turing Machine:

\begin{enumerate}
    \item \textbf{Head:} \\
    A read/write head that moves along the tape one cell at a time.
    It can read the symbol on the tape, write a symbol, or leave the symbol unchanged.

    \item \textbf{State Register:} \\
    Keeps track of the current state of the finite state machine.
    The machine has a finite set of states, one of which is designated as the starting state, and some may be designated as halting states.

    \item \textbf{Transition Function:} \\
    Defines the rules for how the machine behaves based on its current state and the symbol it reads.

    The Transition function specifies the following:
    \begin{itemize}
        \item The new symbol to write on the tape.
        \item The direction to move the head (left, right, or stay in place).
        \item The next state to transition to.
    \end{itemize}
\end{enumerate}

The Turing Machine operates on a tape, which is an infinite one-dimensional strip divided into cells. 
Each cell can hold a single symbol from a finite set of symbols (which includes a blank symbol).

This is the main memory which the Turing Machine operates on and serves as both the input and output medium.

\newpage
\subsubsection{How a Turing Machine works}

To understand how a Turing Machine works, it is beneficial to show an example. The example shows a Turing Machine which has a state machine
with a ``3-state Busy Beaver''. This program shoudl create as many $1$s on the tape as possible whilst also being finite (not executing indefinitely):

\begin{minipage}{0.45\textwidth}
Let's say we have a Turing Machine with the symbols $0,1$ and the blank symbol is $0$. We also define that the starting tape consists of blank symbols.
Now we can define a Turing Machine with a state machine so that it can do something. In this case the state machine has the current state highlighted 
and the transitions are being defined as: \\ \code{<Tape Symbol>|<New Symbol>,<Direction>}. \\ So we define a state machine that is a solution for the ``Busy Beaver'' challenge.
\end{minipage}
\begin{minipage}{0.50\textwidth}
    \centering
    \label{fig:busybeaver-1}
    \begin{turingmachine}{0000000}
        \node[state, current state, initial] (A) {A};
        \node[state, below right of=A,yshift=25] (C) {C};
        \node[state, above right of=C,yshift=-25] (B) {B};

        \node[state, accepting, below of=B,yshift=20] (HALT) {H};

        \draw (A) edge[below] node{0|1,R} (B);
        \draw (A) edge[bend right=15, left] node[xshift=-5]{1|1,L} (C);
        
        \draw (B) edge[loop above] node{1|1,R} (B);
        \draw (B) edge[bend right=15, above] node{0|1,L} (A);

        \draw (C) edge[bend right=15, right] node[xshift=5]{0|1,L} (B);
        \draw (C) edge[bend right=10, above] node[xshift=10]{1|1,R} (HALT);
    \end{turingmachine}
    \vspace{0.2em}
\end{minipage}

\par\noindent\rule{\textwidth}{0.4pt}

\begin{minipage}{0.45\textwidth}
If we now try to determine what the Turing Machine will do next, we can check what the head is currently looking at: $0$. 
This means that the state machine will follow the path to B, write a $1$ onto the tape and move right. 
This means that the Turing Machine will now be in this configuration:
\end{minipage}
\begin{minipage}{0.50\textwidth}
    \centering
    \label{fig:busybeaver-2}
    \begin{turingmachine}{0010000}
        \node[state, initial] (A) {A};
        \node[state, below right of=A,yshift=25] (C) {C};
        \node[state, current state, above right of=C,yshift=-25] (B) {B};

        \node[state, accepting, below of=B,yshift=20] (HALT) {H};

        \draw (A) edge[below] node{0|1,R} (B);
        \draw (A) edge[bend right=15, left] node[xshift=-5]{1|1,L} (C);
        
        \draw (B) edge[loop above] node{1|1,R} (B);
        \draw (B) edge[bend right=15, above] node{0|1,L} (A);

        \draw (C) edge[bend right=15, right] node[xshift=5]{0|1,L} (B);
        \draw (C) edge[bend right=10, above] node[xshift=10]{1|1,R} (HALT);
    \end{turingmachine}
\end{minipage}

Repeating these steps twelve times as these are mostly the same steps:

\begin{minipage}{0.45\textwidth}
After repeating this twelve times we get to this configuration where the Turing Machine is currently
in the state C and the head currently reads the value $1$. This means that the Turing Machine will now go to state H (\code{HALT}).
\end{minipage}
\begin{minipage}{0.50\textwidth}
    \centering
    \label{fig:busybeaver-3}
    \begin{turingmachine}{0111111}
        \node[state, initial] (A) {A};
        \node[state, current state, below right of=A,yshift=25] (C) {C};
        \node[state, above right of=C,yshift=-25] (B) {B};

        \node[state, accepting, below of=B,yshift=20] (HALT) {H};

        \draw (A) edge[below] node{0|1,R} (B);
        \draw (A) edge[bend right=15, left] node[xshift=-5]{1|1,L} (C);
        
        \draw (B) edge[loop above] node{1|1,R} (B);
        \draw (B) edge[bend right=15, above] node{0|1,L} (A);

        \draw (C) edge[bend right=15, right] node[xshift=5]{0|1,L} (B);
        \draw (C) edge[bend right=10, above] node[xshift=10]{1|1,R} (HALT);
    \end{turingmachine}
\end{minipage}

\par\noindent\rule{\textwidth}{0.4pt}

\begin{minipage}{0.45\textwidth}
As the state machine is currently in a accepting state (the double circle) it will stop the Turing Machine and terminate.
It will not go to the next state nor do any modifications on the tape.
\end{minipage}
\begin{minipage}{0.50\textwidth}
\begin{center}
    \label{fig:busybeaver-halt}
    \begin{turingmachine}{1111110}
        \node[state, initial] (A) {A};
        \node[state, below right of=A,yshift=25] (C) {C};
        \node[state, above right of=C,yshift=-25] (B) {B};

        \node[state, current state, accepting, below of=B,yshift=20] (HALT) {H};

        \draw (A) edge[below] node{0|1,R} (B);
        \draw (A) edge[bend right=15, left] node[xshift=-5]{1|1,L} (C);
        
        \draw (B) edge[loop above] node{1|1,R} (B);
        \draw (B) edge[bend right=15, above] node{0|1,L} (A);

        \draw (C) edge[bend right=15, right] node[xshift=5]{0|1,L} (B);
        \draw (C) edge[bend right=10, above] node[xshift=10]{1|1,R} (HALT);
    \end{turingmachine}
\end{center}
\end{minipage}

At the end of the program the Turing Machine has stopped and written $6$ different $1$s onto the tape.

