\subsection{What is Tau?}
Tau is a small programming language that intends to implement a small Turing Machine.

It interprets the code by parsing it, compiling it into a graph and then executing it.

\subsection{Motivation}
This programming language is mainly intended for educational purposes. 
It can be used to showcase the abilities and possibilities of a Turing Machine.

Instead of just learning about Turing Machines in an abstract way, 
people can play around and actually see what the different elements of a Turing Machine do.

\subsection{Turing Machine}
A Turing Machine is a conceptual model of computation that was proposed by the British mathematician and logician Alan Turing in 1936. 
This abstract device helps to explore the boundaries of what can be computed and offers a basis for examining algorithms and computational methods. 
Although it doesn't exist as a tangible machine, the Turing Machine is a cornerstone idea in the fields of computer science and mathematical logic.

The main idea is to build a machine as primitive as possible with the least possible things whilst not sacrificing the possibility
to implement any computer algorithm possible. Because of this primitive nature, many programming languages, to show that they can implement any
computer algorithm, test if they are ``Turing Complete''.

\subsubsection{Components}
The different components of a Turing Machine:

\begin{enumerate}
    \item \textbf{Head:} \\
    A read/write head that moves along the tape one cell at a time.
    It can read the symbol on the tape, write a symbol, or leave the symbol unchanged.

    \item \textbf{State Register:} \\
    Keeps track of the current state of the finite state machine.
    The machine has a finite set of states, one of which is designated as the starting state, and some may be designated as halting states.

    \item \textbf{Transition Function:} \\
    Defines the rules for how the machine behaves based on its current state and the symbol it reads.

    The Transition function specifies the following:
    \begin{itemize}
        \item The new symbol to write on the tape.
        \item The direction to move the head (left, right, or stay in place).
        \item The next state to transition to.
    \end{itemize}
\end{enumerate}

The Turing Machine operates on a tape, which is an infinite one-dimensional strip divided into cells. 
Each cell can hold a single symbol from a finite set of symbols (which includes a blank symbol).

This is the main memory which the Turing Machine operates on and serves as both the input and output medium.

