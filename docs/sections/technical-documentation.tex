The programming language Tau is a very simple language. It only implements a Turing Machine and the interface, which is the programming language.

\subsection{EBNF}
This is how the language is parsed. It does not mean that everything that can be built with this EBNF syntax is also valid.
It is implied that between each token there can be as much whitespace as needed. 
The tokens are being declared with uppercase letters.

\begin{verbatim}
    program = head, body;
    head = { statement };
    body = { state };

    statement = IDENTIFIER, EQUALS, ( list | IDENTIFIER );
    state = IDENTIFIER, CURLY_OPEN, { rule }, CURLY_CLOSE;
    rule = (IDENTIFIER | UNDERSCORE), EQUALS, IDENTIFIER, DIRECTION, IDENTIFIER;

    list = IDENTIFIER, { COMMA, IDENTIFIER };

    IDENTIFIER = character - "_", { character };
    character = "A" .. "Z" | "a" .. "z" | "_";

    EQUALS = "=";
    COMMA = ",";
    UNDERSCORE = "_";

    DIRECTION = "RIGHT" | "R" | "LEFT" | "L" | "STAY" | "S";
\end{verbatim}

\subsection{Linking}
There are two different linking steps:
\begin{enumerate}
    \item Linking for the head where the different statements are being checked and converted.
    \item Linking for the body where the references are being built.
\end{enumerate}

\subsubsection{Head Linking}
The emulator that executes the defined Turing Machine does not use strings. 
This means that all of these strings have to be converted to integers. 
This is being done by using the index of the symbols.

But any statement can be defined in any order, which means that this linking has to be done after
the parsing of the head. This means that this is being done when the divider is being found.

It checks if all the necessary statements are being defined and applies default values if not.
It also checks if the blank is a valid symbol.

\subsubsection{Body Linking}
Any state can be defined in any order. This means that there can be a different state which is being
referenced that will be parsed. This means that the linking is also being done after the parser finds the end
of the file.

The linker also needs the states to be exhaustive. This means that every single possibility is being covered.
For example: If there are the symbols \code{0}, \code{1}, \code{2} and \code{3}, a state like this:
\begin{verbatim}
    TEST {
        1 = 2, RIGHT, TEST
        2 = 2, STAY, HALT
    }
\end{verbatim}
will throw an error as the \code{0} and \code{3} aren't defined. This will happen even if this rule will never be reached
as the linker does not know at this stage if this rule will be accessed.

This can be easily avoided by adding a rule that will be triggered for any other symbol:
\begin{verbatim}
    TEST {
        1 = 2, RIGHT, TEST
        2 = 2, STAY, HALT
        _ = 1, STAY, HALT
    }
\end{verbatim}

The \code{\_} defines that this state should execute this rule in any other scenario on the tape. 
It is like the \code{else} block in many programming languages.

There can't be more than one \code{\_}-rules in one state.
