Some things are not being set up by the program, but rather before executing the program in the command line.

These different settings are flags that can be specified with \code{--\{name\} \{value\}}.

\begin{table}[h]
    \renewcommand\arraystretch{1.5}
    \centering
    \begin{tabular}{c|c|c}
        Name & Value & Meaning \\
        \hline
        \code{view-width} & Number ($\geq 5$) & The elements of the tape that should be shown on each iteration. \\
        \code{max-iter}   & Number ($\geq 1$) & The maximum amount of iterations the Turing Machine should do.
    \end{tabular}
    \caption{The different flags that the Tau programming language has}
\end{table}

So for example if the program is being run with the flags \code{./tau program.tau --max-iter 1} the program will just 
run for one iteration and then terminate.
